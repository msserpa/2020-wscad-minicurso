\begin{resumo}
Tradicionalmente, o aumento de desempenho das aplicações se dava de forma transparente aos programadores através do aumento do paralelismo a nível de instruções e do aumento de frequência dos processadores. Entretanto, esse modelo não se sustenta mais. Para se ganhar desempenho nas arquiteturas modernas, são necessários conhecimentos sobre programação paralela e vetorial. Ambos paradigmas são tratados de forma lateral em cursos de computação, sendo que muitas vezes nem são abordados. Nesse contexto, este capítulo objetiva propiciar um maior entendimento sobre os paradigmas de programação paralela e vetorial, de forma que os participantes aprendam a otimizar adequadamente suas aplicações para arquiteturas modernas. Como plataforma experimental, será utilizado o processador vetorial NEC SX-Aurora TSUBASA. Será enfatizada a importância do processamento vetorial e de matrizes, presente em várias aplicações, tais como de petróleo e na área de previsões climáticas.
\end{resumo}