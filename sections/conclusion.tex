\section{Conclusão} \label{sec:conclusion}

O uso de arquiteturas paralelas e vetoriais é uma solução para incrementar a capacidade de execução, tornando possível a computação eficiente de aplicações com grande demanda de processamento. Para tanto, é preciso fazer uso de interfaces de programação paralela. Neste capítulo foram apresentados detalhes da interface \emph{OpenMP}, a qual é amplamente utilizada em aplicações de alto desempenho para ambientes de memória compartilhada. Com isso, é possível criar aplicações com múltiplas \emph{threads} de forma prática. Desta forma, é possível a solução eficiente de problemas que possuem algum tipo de concorrência. 

Além disso, mostrou-se como utilizar as ferramentas perf e PAPI para analisar o desempenho de aplicações paralelas. Essas ferramentas leem contadores de \textit{hardware} do processador e auxiliam no entendimento de otimizações propostas. Desta forma, como mostrado no estudo de caso de uma aplicação geofísica, é possível maximizar o uso dos recursos computacionais disponíveis nas arquiteturas atuais, com isso, melhorando o desempenho de aplicações sintéticas e reais.

No futuro, planeja-se abordar outras interfaces de programação paralela como Intel TBB, MPI, CUDA, entre outras.

Exercícios e soluções deste capítulo estão disponíveis em:

\url{https://gitlab.com/msserpa/prog-paralela-erad}.
