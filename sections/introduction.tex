\section{Introdução}

A introdução de circuitos integrados, \textit{pipelines}, aumento da frequência das operações, execução fora de ordem e previsão de desvios constituem parte importante das tecnologias introduzidas até o final do século XX. Recentemente, tem crescido a preocupação com o consumo energético, com o objetivo de se atingir a computação em nível \textit{exascale} de forma sustentável. Entretanto, as tecnologias até então desenvolvidas não possibilitam atingir tal fim, devido ao alto custo energético de se aumentar a frequência e estágios de \textit{pipeline}, assim como a chegada nos limites de exploração do paralelismo a nível de instrução~\cite{borkar2011future,coteus2011technologies}. 

Este capítulo envolve o estudo de aspectos relacionados à arquitetura de computadores e a elaboração, execução e teste de programas concorrentes. Neste sentido, pretende-se inicialmente identificar as arquiteturas de \textit{hardware} que existem atualmente e que podem ser utilizadas para a construção de máquinas de alto desempenho. Em um segundo momento, a interface de programação OpenMP será apresentada para ambientes de memória compartilhada. Com base nesta interface, almeja-se elaborar aplicações paralelas otimizadas. 

Para os programas a serem desenvolvidos são disponibilizados códigos fonte sequenciais e paralelos previamente criados e testados.  Os exemplos de código serão introduzido de forma incremental, isto é, variações do mesmo código serão fornecidas para testar aspectos distintos oferecidos pelas interfaces de programação. Por fim, alguns tópicos de testes, depuração e medição de desempenho serão citadas, com o intuito de mostrar como são feitas avaliações de performance de aplicações paralelas.

A estrutura do capítulo é dividida em 7 partes. Inicialmente será apresentada uma introdução sobre as arquiteturas paralelas na Seção~\ref{sec:arquitetura}. Na sequência será explicitado, na Seção~\ref{sec:modelagem}, como pode ser feita a modelagem de aplicações paralelas. A Seção~\ref{sec:openmp} discute como se pode programar paralelamente uma arquitetura de memória compartilhada usando a biblioteca de diretivas OpenMP. Essa seção também aborda a programação vetorial usando instruções SIMD. A avaliação de desempenho com contadores de \textit{hardware} do tipo Linux perf e Performance Application Programming Interface (PAPI) é apresentada na Seção~\ref{sec:contadores}. 
A Seção~\ref{sec:experiments} mostra um estudo de caso sobre o desempenho de uma aplicação de geofísica utilizando os conceitos apresentados no capítulo e, finalmente, a Seção~\ref{sec:conclusion}, traz-se a conclusão, abordando as potencialidades de paralelismo com outras interfaces de programação.